% DOCUMENT CLASS %
\documentclass[letterpaper,twocolumn,openany,nodeprecatedcode]{dndbook}

% IMPORT PACKAGES %
\usepackage[english]{babel}
\usepackage[utf8]{inputenc}
\usepackage[singlelinecheck=false]{caption}
\usepackage{lipsum}
\usepackage{listings}
\usepackage{shortvrb}
\usepackage{stfloats}
\usepackage{xcolor}
\usepackage{graphicx}
\usepackage{wallpaper}

\captionsetup[table]{labelformat=empty,font={sf,sc,bf,},skip=0pt}
\MakeShortVerb{|}

\lstset{%
  basicstyle=\ttfamily,
  language=[LaTeX]{TeX},
  breaklines=true,
}

%%%%%%%%%%
% TITLE PAGE %
\title{Language \& Labyrinths{} \\
\large A Metalinguistic Journey through Erdogic Literature}
\author{Multiplex Void}
\date{2022/04/25}




%%%%%%%%%%%%X%
% BEGIN DOCUMENT %
\begin{document}
\frontmatter
\maketitle
\tableofcontents
\mainmatter

%%%%%%%%%%%%%%%%%%
%%       CHAPTER 1         %%
%%     INTRODUCTION      %%
\part{Introduction}   %%
%%% ABOUT THIS WORK %%%
\chapter{About this Work}
\DndDropCapLine{I}{t seems only appropriate to present } a paper on ergodic literature in ergodic format--it could even be argued 
that such an approach is necessary to properly convey the ideas and concepts contained within the body of this text. 
By its very nature, ergodic literature demands active engagement on the part of its 
|Reader(s)|
    \footnote{The term |Reader| is perhaps too restrictive in describing the role of an "ergodic explorer". 
        It may be more accurate to consider this individual a |user| or |consumer| of ergodic media. 
        Moreover, such  an individual need not act alone. 
        Many ergodic works, such as those exemplified by MUDs, ARGs, and tabletop RPGs, require the participation and cooperation of 
        multiple individuals for storytelling and puzzle-solving purposes.}. 
It requires not only \textit{interpretation} but \textit{reciprocation}. 
It is not meant to be \textit{consumed}, but \textit{explored}. Only through non-trivial effort can one effectively traverse 
such work. \\ 

\begin{DndSidebar}[]{History of Terms}
  The term |Ergotic Literature| was first defined by Espen J. Aarceth in his 1997 book \textit{Cybertext}.
  The term |Cybertext| was first defined in Norbert Wiener's 1948 book \textit{Cybernetics}.
\end{DndSidebar}

%%%%%%%%%%%
%% WHY D&D? %%
\section{Form and Function}
\DndDropCapLine{T}{his work is presented in the style } of a 5e Dungeons and Dragons manual. 
    Why this format, and not another? 
    As the Author of this text, I knew from the beginning that I was going to take an ergodic approach to this assignment. 
    Initially, I had planned on taking on a form inspired by \color{blue}House \color{black} of Leaves. 
    It is a favorite of mine, after all, and has shaped a great deal of my own thoughts and perspectives. 
    However, implementing this proved difficult:  \color{blue}House  \color{black}of Leaves is maddening-- intentionally so.  
    Immersing myself in the text never fails to induce in me an altered state of consciousness--we find ourselves consumed 
    by a kind of madness. This mental state, while phenomenologically interesting, presents significant writing challenges. 
    The thoughts are \textit{too} disorganized, too flighty and entangled and chaotic and confused. 
    I was entirely unable to write in such a way that could properly translate these states and ideas from Author to Reader. 
    Any level of understanding would require significant effort on the part of the Reader. That's asking a bit too much, I think. \\

    Beyond that, there were typesetting concerns-- trying to program all of this in \LaTeX  was simply taking up too much of my time. 
    Searching for an alternative approach, I eventually stumbled upon a DnD 5e  \LaTeX  template--perfect!\\ 

    For one, a DnD manual is a perfect representation of ergodic text. 
    It is not intended to be read like a novel (or a paper), from cover-to-cover in a linear fashion. 
    Its format makes it partiularly suitable as a reference guide. This seemed particularly appropriate for this assignment.  
    What better way to explain and demonstrate ergodicity than by using this format to explore course themes and answer questions of interest?  \\
    
    The role that the |Reader| plays will also shape how they read this document. 
    Are they a (potential) |Player|? A |GM|?
        \footnote{The |Gamemaster|  |(GM)|, plays the role of storyteller; they take on the task of setting up a world and guiding the Players through it.} 
    Perhaps they're curious about tabletop gaming, or perhaps they are unfamiliar with it entirely. 
    Perhaps the Reader is me, the |Author|, as I read and reread and organize and edit this document.
        \footnote{The Author often plays an active role in ergodic literature; they are part of the work as much as the text itself.
        For example, \color{blue} House \color{black} of Leaves is about a nonexistant documentary supposedly recorded 
        by Pulitzer prize winner 
        
        a a book written by the Author Mark Z. Danielewski that is read by the Author, You, 
        that is about a delinquent who 
        ; in otherwords, it is about itself. }
    One particular Reader--our target audience--will play the role of \textit{'professor who is grading a student's assignment according to a specific rubric'.} 
    More specifically, this Reader is known to be a professor who is grading a student's assignment according to a specific rubric who already has familiarity 
    with tabletop RPGs and thus already shares a common cultural background with the author/student.
        \footnote{Hello Dr. Cash!} 
            This gives us the advantage of being able to use tropes and jargon that the Reader is already familiar with and greatly simplifies 
            the effort to communicate effectively.
    
    The purpose of introductory sections such as these are intended to explain the format of this document, its purpose, and how to engage with it. 
    Later sections introduce a framework for the creation of a shared imaginary world.  
    The background and mechanics of this world are explained so that Players can share a common frame of reference for its history and rules. 
    Next, a framework is presented for the creation of a character by a Player. 
    This gives us a framework for imagining how one's character may interact with and act within the shared world.
    In other words, it provides the perfect framework forz  

\chapter{Reference Material}
%% SPELLS %%
\section{Spells}
\DndSpellHeader%
  {Sending}
  {3rd-level Evocation}
  {1 action}
  {Unlimited}
  {V, S, M (a short piece of fine copper wire)}
  {1 Round}
        You send a short message of twenty-five words or less to a creature with which you are familiar. The creature hears the message in its mind, recognizes you as the sender if it knows you, and can answer in a like manner immediately. The spell enables creatures with Intelligence scores of at least 1 to understand the meaning of your message.
        You can send the message across any distance and even to other planes of existence, but if the target is on a different plane than you, there is a 5 percent chance that the message doesn't arrive.


% TEMPLATE %
\part{Template}

%% SECTION %%
\section{Section}
This is a Section

%% SUBSECTION %%
\subsection{Section}
This is a SubSection
%% SUBSUBSECTION %%
\subsubsection{Subsubsection}
Subsubsections are the furthest division of text that still have a block header. Below this level, headers are displayed inline.

%% PARAGRAPH %%
\paragraph{Paragraph}
The paragraph format is seldom used in the core books, but is available if you prefer it to the ``normal'' style.
%% SUBPARAGRAPH %%
\subparagraph{Subparagraph}
The subparagraph format with the paragraph indent is likely going to be more familiar to the reader.


% TEXT  BOXES %
\chapter{Text Boxes}

The module has three environments for setting text apart so that it is drawn to the reader's attention. |DndReadAloud| is used for text that a game master would read aloud.

\begin{DndReadAloud}
  As you approach this module you get a sense that the blood and tears of many generations went into its making. A warm feeling welcomes you as you type your first words.
\end{DndReadAloud}

\section{As an Aside}
The other two environments are the |DndComment| and the |DndSidebar|. The |DndComment| is breakable and can safely be used inline in the text.

\begin{DndComment}{This Is a Comment Box!}
  A |DndComment| is a box for minimal highlighting of text. It lacks the ornamentation of |DndSidebar|, but it can handle being broken over a column.
\end{DndComment}

The |DndSidebar| is not breakable and is best used floated toward a page corner as it is below.

\begin{DndSidebar}[float=!b]{Behold the DndSidebar!}
  The |DndSidebar| is used as a sidebar. It does not break over columns and is best used with a figure environment to float it to one corner of the page where the surrounding text can then flow around it.
\end{DndSidebar}

\section{Tables}
The |DndTable| colors the even rows and is set to the width of a line by default.

\begin{DndTable}[header=Nice Table]{XX}
    Table head  & Table head \\
    Some value  & Some value \\
    Some value  & Some value \\
    Some value  & Some value
\end{DndTable}

\chapter{Monsters and NPCs}

% Monster stat block
\begin{DndMonster}[float*=b,width=\textwidth + 8pt]{Ludwig Wittgenstein}
  \begin{multicols}{2}
    \DndMonsterType{Medium aberration (metasyntactic variable), neutral evil}

    % If you want to use commas in the key values, enclose the values in braces.
    \DndMonsterBasics[
        armor-class = {9 (12 with \emph{mage armor})},
        hit-points  = {\DndDice{3d8 + 3}},
        speed       = {30 ft., fly 30 ft.},
      ]

    \DndMonsterAbilityScores[
        str = 12,
        dex = 8,
        con = 13,
        int = 10,
        wis = 14,
        cha = 15,
      ]

    \DndMonsterDetails[
        %saving-throws = {Str +0, Dex +0, Con +0, Int +0, Wis +0, Cha +0},
        %skills = {Acrobatics +0, Animal Handling +0, Arcana +0, Athletics +0, Deception +0, History +0, Insight +0, Intimidation +0, Investigation +0, Medicine +0, Nature +0, Perception +0, Performance +0, Persuasion +0, Religion +0, Sleight of Hand +0, Stealth +0, Survival +0},
        %damage-vulnerabilities = {cold},
        %damage-resistances = {bludgeoning, piercing, and slashing from nonmagical attacks},
        %damage-immunities = {poison},
        %condition-immunities = {poisoned},
        senses = {darkvision 60 ft., passive Perception 10},
        languages = {Common, Goblin, Undercommon},
        challenge = 1,
      ]
    % Traits
    \DndMonsterAction{Innate Spellcasting}
    Foo's spellcasting ability is Charisma (spell save DC 12, +4 to hit with spell attacks). It can innately cast the following spells, requiring no material components:
    \begin{DndMonsterSpells}
      \DndInnateSpellLevel{misty step}
      \DndInnateSpellLevel[3]{fog cloud, rope trick}
      \DndInnateSpellLevel[1]{identify}
    \end{DndMonsterSpells}

    \DndMonsterAction{Spellcasting}
    Foo is a 2nd-level spellcaster. Its spellcasting ability is Charisma (spell save DC 12, +4 to hit with spell attacks). It has the following sorcerer spells prepared:
    \begin{DndMonsterSpells}
      \DndMonsterSpellLevel{blade ward, fire bolt, light, shocking grasp}
      \DndMonsterSpellLevel[1][3]{burning hands, mage armor, shield}
    \end{DndMonsterSpells}

    \DndMonsterSection{Actions}
    \DndMonsterAction{Multiattack}
    The foo makes two melee attacks.

    %Default values are shown commented out
    \DndMonsterAttack[
      name=Dagger,
      %distance=both, % valid options are in the set {both,melee,ranged},
      %type=weapon, %valid options are in the set {weapon,spell}
      mod=+3,
      %reach=5,
      %range=20/60,
      %targets=one target,
      dmg=\DndDice{1d4+1},
      dmg-type=piercing,
      %plus-dmg=,
      %plus-dmg-type=,
      %or-dmg=,
      %or-dmg-when=,
      %extra=,
    ]

    %\DndMonsterMelee calls \DndMonsterAttack with the melee option
    \DndMonsterMelee[
      name=Flame Tongue Longsword,
      mod=+3,
      %reach=5,
      %targets=one target,
      dmg=\DndDice{1d8+1},
      dmg-type=slashing,
      plus-dmg=\DndDice{2d6},
      plus-dmg-type=fire,
      or-dmg=\DndDice{1d10+1},
      or-dmg-when=if used with two hands,
      %extra=,
    ]

    %\DndMonsterRanged calls \DndMonsterAttack with the ranged option
    \DndMonsterRanged[
      name=Assassin's Light Crossbow,
      mod=+1,
      range=80/320,
      dmg=\DndDice{1d8},
      dmg-type=piercing,
      %plus-dmg=,
      %plus-dmg-type=,
      %or-dmg=,
      %or-dmg-when=,
      extra={, and the target must make a DC 15 Constitution saving throw, taking 24 (7d6) poison damage on a failed save, or half as much damage on a successful one}
    ]

    % Legendary Actions
    \DndMonsterSection{Legendary Actions}
    The foo can take 3 legendary actions, choosing from the options below. Only one legendary action option can be used at a time and only at the end of another creature's turn. The foo regains spent legendary actions at the start of its turn.

    \begin{DndMonsterLegendaryActions}
      \DndMonsterLegendaryAction{Move}{The foo moves up to its speed.}
      \DndMonsterLegendaryAction{Dagger Attack}{The foo makes a dagger attack.}
      \DndMonsterLegendaryAction{Create Contract (Costs 3 Actions)}{The foo presents a contract in a language it knows and waves it in the face of a creature within 10 feet. The creature must make a DC 10 Intelligence saving throw. On a failure, the creature is incapacitated until the start of the foo's next turn. A creature who cannot read the language in which the contract is written has advantage on this saving throw.}
    \end{DndMonsterLegendaryActions}
  \end{multicols}
\end{DndMonster}

The |DndMonster| environment is used to typeset monster and NPC stat blocks. The module supplies many functions to easily typeset the contents of the stat block



%% Glossary %%
\part{Glossary}
\DndDropCapLine{W}{hat is language \textit{for?} } a paper on ergodic literature in an ergodic format.  Ergodic literature, after all, demands active engagement on the part of its player(s).\footnote{ Throughout this text, we will refer to someone who engages with an ergodic work as a |player|, as opposed to a |reader|. See |Cybertext| .}
est

%----------------------------------------------------------------------------------------
%	BIBLIOGRAPHY
%----------------------------------------------------------------------------------------



\end{document}
