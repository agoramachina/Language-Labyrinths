
\usepackage(glossaries)
\makeglossaries

\newglossaryentry{erdolit}
{
    name=erdogic literature,
    description={A text that requires non-trivial effort to navigate. The form of the text is considered an essential element of the work itself.}
}

\newacronym{gm}{GM}{Game Master. In the original DnD rulebooks, this individual was referred to as the \textit{Dungeon Master}. With the growing number of tabletop rpgs since the release of Dnd 1e, a popular generic term is \textit{Game Master}. However, this term is considered problematic by some; proposed alternative titles include \textit{Storyteller}, \textit{Moderator}, and \textit{Host}, among others.}
\newacronym{DnD}{Dungeons and Dragons}{Can also be written \textit{'D\&D'}}
\newacronym{RPG}{Role Playing Game}{In an RPG, players take on the role of a particular character who is acting in a certain rule according to certain situations.}

\newacronym{MUD}{Multi-user Dungeon}{e.g. LegendMUD, MicroMUSE, Hourglass. MUDs are typically (though not always) text based.}

\newacronym{ARG}{Alternate Reality Game}
